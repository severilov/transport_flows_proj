\documentclass{article}
\usepackage[utf8]{inputenc}
\usepackage[russian]{babel}
\usepackage{graphicx}


\title{Транспортная теория}
% \author{Ревар Владимир}
\date{December 2021}

\begin{document}

\maketitle

\section{Введение}
\subsection{Применение в реальной жизни}
Транспортные системы городов и их ареалы являются одним из важных
факторов, который влияет на социально-экономическое развитие страны.
Совершенствование транспортной сети повышает качество жизни горожан,
увеличивает рост занятости, укрепляет бюджет города, развивает бизнес и
привлекает инвестиции. \\
Дороги являются "каркасом" страны (города), с помощью которого
соединяются между собой все части страны (города). Поэтому в
идеальной модели рыночной экономики улично-дорожные сети, в частности
и городские, должны обеспечивать высокую мобильность людей и
беспрепятственную перевозку товаров, так как состояние транспортной сети
напрямую влияет на состояние экономики в целом.

Для определения потоков и загрузки элементов сети сегодня существует
много математических моделей. Но основной проблемой в большинстве
данных моделей является необходимость информации о передвигающихся
по городу индивидуумах. Эта информация представляется в виде матрицы
корреспонденций. \\

\textbf{Матрица корреспонденций} - информация о том, откуда, сколько и куда людей направляются в единицу времени.

Если матрица корреспонденций построена, то на её основе можно:
\begin{itemize}
    \item составить наиболее точное расписание движения общественного
транспорта
    \item определить загрузки элементов улично-дорожной сети
    \item определить главные пассажирообразующие пункты
    \item оценить количество перевозимых пассажиров по типам пассажиров, по
видам транспорта, маршрутам и направлениям
    \item оценить интенсивность пассажиропотоков между различными
пунктами
\end{itemize}

\subsection{Математические понятия}

\textbf{Равновесный принцип Нэша–Вардропа}. Для объяснения ниже будет приведен пример
Рассмотри следующую транспортную систему:

\includegraphics[scale=0.5]{traffic_jam_3_600.jpg}

Из 1 в 4 в единицу времени выезжает 6 автомобилей.\\

Буквой $y$ обозначается количество автомобилей, проезжающих по данному ребру в единицу времени (поток), формулы над ребрами — веса ребер (время проезда ребра в минутах в зависимости от величины потока автомобилей по ребру y).\\

$f_e$ - поток по ребру $e$, где $e \in E = {(12), (24), (13), (34)}$ - множество всех ребер.

Обозначим за $y_{124}, y_{134}$ - потоки на путях $124$ и $134$ соответсвенно.
Аналогично ребрам, вводится множество путей $P = {(124), (134)}$.

$y_p$ - поток по пути $p \in P$.
В данном примере $$y_{124} = f_{12} = f_{24}, y_{134} = f_{13} = f_{34}$$\\

В лекциях присутствует следующее понятие:
$d_{14}$ - \textbf{корреспонденция} - сколько людей в единицу времени хочет перемещаться из $1$ в $4$.

$y_{124} + y_{134} = d = 6$ \\
Итого, можем записать следующее равенство: $$f = \Theta x$$
Потоки по ребрам связаны с потоками по путям.

Условие равновесия: все используемые пути должны иметь одинаковую «длину». Всего путей, ведущих из 1 в 4 три: 1–3–4, 1–3–2–4, 1–2–4.\\

Легко проверить, что если эти 6 автомобилей в одинаковых пропорциях распределятся по всем этим путям (по 2 автомобиля в единицу времени на каждый путь), то время прохождения каждого пути («длина» пути) будет равна 92 минутам (10 · (2 + 2) + (50 + 2) = 10 · (2 + 2) + (10 + 2) + 10 · (2 + 2) = (50 + 2) + 10 · (2 + 2)). Это и будет единственным равновесием (Нэша—Вардропа) в транспортной сети. То есть от такой конфигурации никому из водителей не выгодно отклоняться при условии, что остальные водители не меняют свой выбор.\\

\textbf{Транспортная система}
В работе рассматривается замкнутая транспортная система, описываемая графом $G=<V,E>$, где $V$- множество вершин ($|V|=n$), а $E$ - множество ребер ($|E|=m$)

Обозначим через $\tau_e(f_e)$ - функцию затрат (например, временных) на проезд по ребру $e$, если поток автомобилей на этом участке $f_e$.

$\tau_e(f_e) = \overline t_e \cdot (1+ k(\frac{f_e}{\overline f_e}))^{\frac{1}{\mu}}$, где
\begin{itemize}
    \item $\overline t_e$ - время проезда ребра $e$ по свободной дороге
    \item $\overline f_e$ - максимальная пропускная способность ребра [авт/час]
    \item $k$ - какой-то небольшой коэффициент
    \item $\frac{1}{\mu} = 4$ (обычно берут $4$) - показатель степени
\end{itemize}
Вообще, данная функция относится к классу функций $BPR$.
Также допускается $\mu -> 0+$ - моедль стабильной динамики. \\

Введем $t_e$ - временные затраты на прохождение ребра $e$. $t_e = \tau_e(f_e)$. По затратам $t = \{t_e\}_{e \in E}$ можно определить затраты на перемещение из источника $i$ в сток $j$ по кратчайшему пути:
$$T_{ij} = \min_{p \in P_{ij}} \sum_{e \in E} \delta_{ep}t_e$$, где:
\begin{itemize}
    \item $p$ - путь (без самопересечений, циклов) на графе (набор ребер)
    \item $P_{ij}$ - множество всевозможный путей на графе, стартующих в $i$ и заканчивающих в $j$
    \item $\delta_{ij}$ - дельта функция, показывающая, принадлежит ли ребро $e$ пути $p$
    \item \textbf{Корреспонденция} .Часть вершин $O \subset V (origin)$ являются источниками корреспонденций, а часть стоками корреспонденций $D \subset V (destination)$. В примере выше, мы рассмотрели корреспонденцию, но в общем случае, корреспонденций может быть несколько, обозначаются они следующим образом $d_{ij} = (i,j) \in (OD) $. Если говорить более точно, то вводится множество пар (источник, сток) корреспонденций $OD \subset V * V$. Не ограничивая общности, будем далее считать, что $\sum_{(i,j)\in(OD)} d_{(i,j)} = 1$. В общем случае корреспонденции – не известны! Однако известны (заданы) характеристики источников и стоков корреспонденций. То есть известны величины $\{l_i\}_{i \in O}, \{w_j\}_{j \in D}$
    $$\sum_{j: (i,j) \in (OD)} d_{ij} = l_i,  \sum_{i: (i,j) \in (OD)} d_{ij} = w_j \;\;\;\; (1)$$. 
    
    Кроме того "сколько вышло, столько и пришло" $$\sum_{i \in O} l_i = \sum_{j \ in D} w_j = 1$$. 
    
    Условие $(1)$ для краткости будем записывать $d \in (l,w)$
    
    Кроме этого, в примере у нас присутствовала матрица $\Theta$. Эта матрица имеет следующий вид: $\Theta = ||\delta_{ep}||_{e\in E, p \in P}$
\end{itemize} \\

\subsection{Полезные уравнения:}
$$1. f_e = \sum_{p \in P} \delta_{p} x_p$$ - поток на ребре $e$ равен сумме потоков по тем путям, в которые входит это ребро.
$$2. \sum_{p \in P_w} x_p = d_w, w \in W$$ \\

\textbf{Затраты на пути $p$:} \\
$T_p = \sum_{e \in E} \delta_{ep} \tau_e(f_e(x))$ - суммируем по ребрам, которые входят в данный путь.

Что означает, что $x*$ - равновесие?

$\forall w \in W, \forall p \in P_w$: $$X_p^* \cdot (T_p(x^*) - \min_{q \in P_w}T_q(x^*))=0  \;\;\;\; (2)$$  \\
$(2)$ - условие комплиментарности. Также можно считать переформулировкой понятия равновесия по Нэшу. \\
$p \in P_w$ - пути, отвечающие корреспонденции $w$

\section{Энтропийная модель расчета матрицы корреспонденций}
Под энтропийной моделью расчета матрицы корреспонденций $d(T)$ понимается определенный способ вычисления набора корреспонденций ${d_{ij}}_{(i,j)\in W}$ по известной матрице
затрат $\{T_{ij}\}_{(i,j) \in W}$. Этот способ заключается в решении задачи энтропийно-линейного программирования, которую можно понимать, как энтропийно-регуляризованную транспортную задачу:
$$\min_{d \in (l,w); d \ge 0} \sum_{(i,j) \in OD} d_{ij}T_{ij} + \gamma \sum_{(i,j) \in OD} d_{ij}lnd_{ij} \;\;\;\; (3)$$,
где параметр $\gamma > 0$ считается известным.

\section{Модели равновесного распределения транспортных потоков по путям:}
Матрица корреспонденций ${d_{ij}}_{(i,j)}\in OD$ порождает (вообще говоря, неоднозначно) некий вектор распределения потоков по путям x. Неоднозначность заключается в том, что балансовые ограничения, которые возникают на $$x \in X(d): x \ge 0 : \forall(i, j) \in OD -> \sum_{p \in P_{ij}} x_p = d_{ij}$$
как правило, не определяют вектор $x$ однозначно. Вектор $x$, в свою очередь, пораждает вектор потоков на ребрах, $f = \Theta x$, который, в свою очередь, порождает вектор (временных) затрат на ребрах $t(f) = {\tau_e(f_e)}_{e \in E}$. На основе последнего вектора уже можно рассчитать матрицу затрат на кратчайших путях $T(t) = \{T_{ij}(t)_{(i,j)\in OD}\}$. Собственно, модель равновесного распределения потоков это формализация принципа Нэша–Вардропа о том, что в равновесии каждый водитель выбирает для себя кратчайший путь. Другими словами, если для заданной корреспонденции $(i, j) \in OD$ известно, что (условие комплиментарности)
$$x_{p^{'}} > 0, p^{'} \in P_{ij} -> T_{ij}(t) = \min_{p \in P_{ij}} \sum_{e\in E}\delta_{ep}t_e = \sum_{e \in E} \delta_{ep^{'}}t_e $$
Решение задачи дает модель вычисления вектора потока на ребрах при заданной матрице корреспонденций $f(d)$

Задача поиска равновесия сводится, таким образом, к поиску такого вектора $x \in X(d)$, который бы порождал такие затраты $T := T(t(f(x)))$, что выполянется условие комплиментарности. В написанном выше виде искать равновесный вектор $x \in X(d)$ представляется сложной задачей, сводящейся к решению системы нелинейных уравнений. Однако, в данном случае (рассматривается потенциальная игра загрузки) можно свести поиск равновесия к решению задачи выпуклой оптимизации
$$ \min_{(f,x): f = \Theta x; x \in X(d)} \sum_{e \in E} \int_0^{f_e}\tau_e(z)dz \;\;\;\; (4)$$
Решение задачи дает модель вычисления вектора потока на ребрах при заданной матрице корреспонденций $f(d)$

\section{Двухстадийная модель:}
Выше были описаны две модели. В первой (расчёт матрицы корреспонденций) на вход подается матрица затрат $T$, а на выходе получается матрица корреспонденций $d(T)$. Во второй модели наоборот, на вход подается матрица корреспонденций $d(T)$, а на выходе рассчитывается матрица затрат $T(d) = T(t(f(d)))$.

Под равновесием в двухстадийной транспортной модели понимается такая пара $(f, d)$, что
$$d = d(T(f(t))), f = f(d) \;\;\;\; (5)$$
то есть $(f, d)$ – есть неподвижная точка описанных двух блоков моделей. Собственно, часто на практике так и ищут равновесие последовательно (друг за другом) прогоняя описанные два блока. Но присутствует проблема, что нет никаких теоретических гарантий, что последовательная прогонка будет сходиться к неподвижной точке.
Даже если наблюдается сходимость, то непонятно, насколько эта сходимость может быть быстрой, и лучший ли это способ  (простая прогонка) численного решения (5)?

\textbf{Теорема 1:} Задача поиска неподвижной точки (5) сводится к задаче выпуклой оптимизации:
$$ \min_{(f,x,d): f=\Theta x; x \in X(d); d \in (l,w); d \ge 0}\sum_{e \in E} \int_0^{f_e} \tau_e(z)dz + \gamma \sum_{(i,j) \in OD} d_{ij} ln \; d_{ij} \;\;\;\; (6)$$.
Смысл в том, что теперь это задача выпуклой оптимизации, которую можно решать оптимальным по скорости (глобально сходящимися) алгоритмами.
\textbf{Доказательство присутствует в самой статье, выписывать его нет смысла.}

Заметим, что выписанная задача $(6)$, как задача оптимизации относительно $d$ при «заморожоженных» $(f, x)$, совпадает с задачей $(3)$ и, наоборот, при «замороженном» $d$ задача $(6)$, как задача оптимизации относительно $(f, x)$, совпадает с задачей $(4)$.

Таким образом, если удалось найти такую задачу, решение которой одновременно дает нужные нам связи переменных, описываемые формулой $(5)$, то это и означает, что нам удалось свести поиск неподвижной точки сложного нелинейного отображения (которое не удается выписать аналитически) к явно выписанной задаче оптимизации $(6)$. Ометим, что решение этой выпуклой задачи оптимизации по сложности сопоставимо с решением задачи $(4)$, что будет пояснено в следующем разделе.

\section{Переход к двойственной задаче}
задача выпуклой оптимизации (6) можно переписать эквивалентным седловым образом, введя двойственные переменные $t = \{te\}_{e \inE}$,  которые имеют естественную интерпретацию вектора потоков на ребрах,
$$\min_{d\in(l,w); d \ge 0} \max_{t_e \in dom \; \sigma_e^*, e\in E} \left\{\sum_{(i,j) \in OD}d_{ij}T_{ij}(t) - \sum_{e \in E} \sigma_e^*(t_e) + \gamma \sum_{(ij) \in OD} d_{ij} ln \; d_{ij} \right\} \;\;\;\; (**),$$
где $\sigma_e(f_e) = \int_0^{f_e}\tau_e(z)dz$, а $\sigma_e^*(t_e) = max_{f_e\ge 0 }\{f_et_e - \sigma_e(f_e) \}$ - сопряженная функция.

Уравнение $(**)$ удобнее переписать следующим образом:
$$\max_{t_e \in dom \; \sigma_e^*, e \in E} \;\; \min_{d\in (l,w); \sum_{(i,j) \in OD};d_{ij} = 1, d \ge 0} \left\{\sum_{(i,j) \in OD} d_{ij}T_{ij}(t) + \gamma \sum_{(i,j) \in OD}d_{ij} ln\;d_{ij} \right\} - \sum_{e \in E}\sigma_e^*(t_e)$$
Вспомогательную задачу минимизации можно представить через двойственную к ней:
$$\max_{t_e \in dom \sigma_e^*; e \in E; (\lambda, \mu)}\;\; - \gamma \; ln \left( \sum_{(i,j) \in OD} exp\left(\frac{-T_{ij}(t) + \lambda_i + \mu_j}{\gamma} \right) \right) + <l,\lambda> +<w, \mu> - \sum_{e \in E}\sigma_e^*(t_e) =$$
$$-\min_{t_e \in dom \sigma_e^*; e \in E; (\lambda, \mu}\;\; \gamma \; ln \left( \sum_{(i,j) \in OD} exp\left(\frac{-T_{ij}(t) + \lambda_i + \mu_j}{\gamma} \right) \right) - <l,\lambda> -<w, \mu> + \sum_{e \in E}\sigma_e^*(t_e) \;\;\; (7)$$

Обратим внимание, что добавленное по $d$ ограничение $ \sum_{(i,j) \in OD} d_{ij} = 1$ тавтологично, поскольку следует из $d \in (l, w)$. Тем не менее, удобнее его добавить, чтобы при взятии $min$ получалась равномерно гладкая функция (типа softmax), а не сумма экспонент, имеющая неограниченные константы гладкости. Множители $\lambda$ и $\mu$ являются двойственными множителями (множителями Лагранжа) к ограничениям $d \in (l, w)$, которые заносятся в функционал (ограничения $\sum_{(i,j) \in OD} d_{ij} = 1; d \ge 0$ не заносятся в функционал). Заметим, что если $(t, \lambda, \mu)$ – решение задачи (7), то
$(t, \lambda +  (C_{\lambda}, \ldots,C_{\lambda})^{T}, \mu + (C_{\mu}, \ldots, C_{\mu})^{T})$ - также будет решением, т.е. решение $(7)$ не единственное.  Заметим также, что зная $(\lambda, \mu)$, можно посчитать матрицу корреспонденци:
$$d_{ij}(\lambda, \mu) = \frac{exp\left(\frac{-T_{ij}(t) + \lambda_i + \mu_j}{\gamma} \right)}{\sum_{(k,l) \in OD} exp \left(\frac{-T_{kl}(t) + \lambda_k + \mu_l}{\gamma} \right)} \;\;\; (8)$$


Для решения задачи выпуклой оптимизации (но, вообще говоря, не гладкой, поскольку функции $T_{ij}(t)$ – негладкие) можно использовать субградиентные методы. А именно, субградиент (далее обозначаем (супер-)субградиент таким же символом, как и градиент $\nabla$) целевого функционала по $t$ (стоящего под минимумом) (7) можно посчитать по формуле Демьянова–Данскин
$$\sum_{(o,j) \in OD} d_{ij}(\lambda, \mu) \nabla T_{ij}(t) + f = \sum_{(i,j) \in OD} d_{ij}(\lambda, \mu) \nabla T_{ij}(t) + \left( \{ \tau_e^{-1}(t_e) \}_{e \in E} \right)^{T} \;\;\; (9)$$,
где $\tau_e^{-1} - $ обратная функция к $\tau_e$.

Примечательно, что отличие формулы $(9)$ от ее аналога, который можно получить, решая задачу $(4)$ посредством перехода к двойственной задач только в том, что $d_{ij} = d_{ij}(\lambda, \mu)$, где $(\lambda, \mu)$ определяются из решения задач 
$$\min_{(\lambda, \mu)} D(t, \lambda, \mu) := \gamma ln \; \left(\sum_{(i,j) \in OD} exp \left( \frac{-T_{ij}(T) + \lambda_i +\mu_j}{\gamma}\right) \right) - <l, \lambda> - <w, \mu> \;\;\; (10)$$

Важное наблюдение, заключается в том, что решать задачу $(7)$ выгоднее не как задачу оптимизации по переменным $(t, \lambda, \mu)$, а как задачу только по переменной $t$, в то время как переменные $(\lambda, \mu)$ лишь используются для подсчета субградиента целевого функционала по формуле $(9)$ с $d_{ij}(\lambda, \mu)$ рассчитываемым по формуле $(8)$, в которой
$$(\lambda, \mu) := (\lambda (t), \mu (t)) \in arg\min_{(\lambda, \mu)}D(t, \lambda, \mu) \;\;\; (11)$$
определяются как решение задачи $(10)$. 2). Заметим, что сложность вычисления $\nabla T_{ij} (t)$ оптимальным алгоритмом Дейкстры будет сопоставима со сложностью вычисления (с нужной точностью) матриц $d_{ij}(\lambda (t), \mu (t))$.
Таким образом, получается, что сложность вычисления субградиента для двойственной задачи к $(4)$ и для задачи $(7)$ сопоставимы. При этом, свойства гладкости (определяющие скорость сходимости используемых методов) целевого функционала в задаче $(7)$ при переходе от оптимизации в пространстве $(t, \lambda, \mu)$ к оптимизации по переменной $t$ могут только улучшиться (в любом случае не ухудшиться)
\end{document}
